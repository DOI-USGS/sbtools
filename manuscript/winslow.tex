% !TeX root = RJwrapper.tex
\title{\pkg{sbtools}: A package connecting R to data for collaborative online research}
\author{by Luke A Winslow, Scott Chamberlain, Alison Appling and Jordan Read}

\maketitle

\abstract{
An abstract of less than 150 words.
}

Introductory section which may include references in parentheses
\citep{R}, or cite a reference such as \citet{R} in the text.

\section{Introduction}

\begin{itemize}
	\item{Creating powerful and useful tools to open and accelerate science an important issue of our time}
	\item{One rapidly emerging area is reproducibile research and data archiving}
	\item{These goals primarily focus creating more useful, open, and accessible products of research}
	\item{But data- and code-intensive collaborative projects increasingly need collaborative solutions for data storage, sharing, and updating throughout the project lifecycle}
	\item{Active collaborations have different needs}
	\item{Need large-file storage and access, Mixture of public and controlled access, programmatic access to retrieving data and updating results}
	\item{With sbtools, we aim to fill these needs.}
	\item{By building on the web API's of ScienceBase, we }
	
\end{itemize}


\section{USGS ScienceBase}

\begin{itemize}
	\item{What is ScienceBase \url{https://www.sciencebase.gov}}
	\item{Project build for USGS and collaborators}
	\item{Large, flexible, online repository for scientific data}
	\item{Fundamentally hierarchical in design, build on uniquely identified items and children}
	\item{Items have rich metadata functionality that improves data discovery and sharing}
	\item{Build with first-class REST API in mind. All functionality available via web service}
	\item{Available to USGS (DOI?) employees and project collaborators}
\end{itemize}

\section{The \pkg{sbtools} package}
\begin{itemize}
	\item{Goal to allow programmatic access to core SB functionality}
	\item{Harmonize the SB data structures with common R interfaces}
	\item{Lightweight R package that can be quickly added to existing projects and shared broadly}
\end{itemize}

\subsection{Search API}
\begin{itemize}
	\item{Go over general search API}
	\item{Free text search}
	\item{dateRange query}
	\item{Spatial query}
	\item{Hierarchical SB queries (sub-items)}
	\item{Combination of above}
\end{itemize}

\subsection{Data access API}
\begin{itemize}
	\item{Basic item metadata access}
	\item{Attached file listing and downloading}
	\item{Accessing special SB extensions}
	\item{Hierarchy traversal}
\end{itemize}

\subsection{ScienceBase authenticated}
\begin{itemize}
	\item{How authentication changes SB}
	\item{Authentication in sbtools}
	\item{Authentication persistence and use throughout package}
	\item{Examples of authentication and other auth-related functions (check and such)}
\end{itemize}


\subsection{Data editing and upload API}
This will be an extensive section
\begin{itemize}
	\item{Item creation}
	\item{Item deletion}
	\item{Attached file upload, replacement and deletion}
	\item{}
\end{itemize}

\subsection{SB Item identifiers}
\begin{itemize}
	\item{Functionality of item identifiers}
	\item{Example of identifier use}
	\item{How identifiers can be useful for projects}
\end{itemize}

\section{Summary}
\begin{itemize}
	\item{Building tools to improve data access and handling for collaborative projects will be important in the future}
	\item{sbtools gives users direct access to a large collection of USGS data}
	\item{It also links online data storage directly to R workflows}
	\item{With the popularity and power of the R universe growing, opening access to data, along with code online will be important}
\end{itemize}


\bibliography{winslow}

\address{Luke A Winslow\\
  U.S. Geological Survey Center for Integrated Data Analytics\\
  Middleton, Wisconsin\\
  USA\\}
\email{lwinslow@usgs.gov}

\address{Scott Chamberlain\\
  Affiliation\\
  Address\\
  Country\\}
\email{author2@work}

\address{Alison Appling\\
  Affiliation\\
  Address\\
  Country\\}
\email{author3@work}

\address{Jordan S Read\\
  U.S. Geological Survey Center for Integrated Data Analytics\\
  Middleton, Wisconsin\\
  USA\\}
\email{jread@usgs.gov}


%some figure code example
%This section may contain a figure such as Figure~\ref{figure:rlogo}.
% \begin{figure}[htbp]
%   \centering
%   \includegraphics{Rlogo}
%   \caption{The logo of R.}
%   \label{figure:rlogo}
% \end{figure}

%code example
% \begin{example}
  % x <- 1:10
  % result <- myFunction(x)
% \end{example}
