\section{Introduction}

Cloud data storage platforms can be a powerful tool for research collaboration,
distributed computing, and data publication. However, while browser-based graphical
interfaces make these platforms accessible to a wide audience, there is also
a strong need for scripted data handling so that researchers can rigorously
document data provenance and create reproducible analyses.
% As computing tools become more powerful, leveraging that power to
% create useful tools to open and accelerate science is an important issue
% of our time. One area of rapid development is the archiving, sharing,
% and integration of research data into reproducible workflows.
For example,
Figshare (ref), DataOne (ref), and Dataverse (ref) are all data research data
sharing platforms gaining use in different research fields. For each of these
platforms, the community has released (\cite{rfigshare}, \cite{dvn}) or is
developing (\cite{dataone}) R packages to streamline the storage and access
to archived data.

These existing projects primarily focus on creating more useful, open, and
accessible research products. But data- and code-intensive collaborative projects
increasingly need collaborative solutions for data storage, sharing, and updating
throughout the full lifecycle of the project. Furthermore, many of these data
archive platforms focus on academic research projects and do not support some
of the unique needs of federal research.
% what are those unique needs?

To address future scientific data sharing and archival challenges, the USGS
created ScienceBase (https://www.sciencebase.gov/catalog/). This platform is
designed to support
the full project data lifecycle and has seen rapid adoption with USGS
researchers and collaborators. ScienceBase supports the storage and access
of large files and datasets. It allows data to be stored with a mixture of
public and controlled access and has been designed from the beginning with
first-class RESTful web interfaces to support robust API access. It also allows
a seamless transition from project development to data publication, with
a focus on public data access once data have been released.

To expand the usefulness of the ScienceBase platform, we have designed and
implemented an R interface to ScienceBase called \pkg{sbtools} (\cite{sbtools}).
This interface provides scripted R access to ScienceBase to manage metadata and
data files, to search the catalog of datasets, and to view and modify data in
formats familiar to R users.
Here we describe unique and useful features of ScienceBase, and how we have
implemented the R interface to make them accessible and useful from R.
