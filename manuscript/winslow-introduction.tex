
\section{Introduction}

Cloud data storage platforms can be a powerful tool for research collaboration,
distributed computing, and data publication. However, while browser-based graphical
interfaces make these platforms accessible to a wide audience, there is also
a need for scripted data access and manipulation so that researchers can capture 
data provenance and create reproducible analyses. For example,
Figshare \citep{figshare}, DataOne \citep{dataone}, Dataverse \citep{dataverse} 
and CKAN \citep{ckan} are all research data
sharing platforms that are gaining use in different research fields. For each of these
platforms, the community has released \citep{rfigshare, dvn, ckanr} or is
developing \citep{dataonepkg} R packages to streamline the storage and access
to archived data. 

These existing projects primarily focus on creating more useful, open, and
accessible end products of research, but data- and code-intensive collaborative projects
increasingly need collaborative solutions for data storage, sharing, and updating
not just at the end of the project, but throughout the full project lifecycle. 
Github is increasingly used to collaborate around data products \citep{GandrudGithub}, 
but does not scale well to the distribution of large datasets \citep{Delcambre2013} and does not
include metadata or queries beyond free-text search. Furthermore, 
most data archive platforms like DataOne and Dataverse 
focus on academic research projects and do not support some
of the unique needs of federal research. For example, many government institutions
require an archived copy of released data to be stored and available through federally
operated websites. Third party storage providers are insufficient.

To address future scientific data sharing and archival challenges, the U.S. 
Geological Survey (USGS) created ScienceBase (\url{https://www.sciencebase.gov/}). 
This platform is designed to support the full project data lifecycle and has seen 
rapid adoption with USGS researchers and collaborators. ScienceBase supports the 
storage and access of large files and datasets. It allows data to be stored 
with a user-configurable mixture of 
public and authenticated access and has been designed from the beginning with
first-class RESTful web interfaces to support robust API access. Items on ScienceBase 
can have hierarchical relationships, facilitating the organization of complex or related
data. It also supports a seamless transition from project development to data publication, 
focusing on searchable, accessible, well described datasets for public use and citation.

To expand the usefulness of the ScienceBase platform directly to R workflows, 
we have designed and
implemented an R interface to ScienceBase called \pkg{sbtools}.
This interface provides scripted R access to ScienceBase to manage metadata and
data files, to search the catalog of datasets, and to view and modify data in
formats familiar to R users.
Here we describe several features of ScienceBase, and how we have
implemented the R interface to make them accessible and useful from R.
