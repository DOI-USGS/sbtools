\section{Introduction}

As computing tools become more powerful, leveraging that power to 
create useful tools to open and accelerate science an important issue 
of our time. One area of rapid development is the archiving, sharing, 
and integration of reserach data into reproducible workflows. For example,
figshare (ref), DataOne (ref), and the Dataverse (ref) are all different data research data
sharing platforms gaining use in different research fields. For each of these 
platforms, the community has released (\pkg{rfigshare}, \pkg{dvn}) or is 
developing (\pkg{rdataone}) R packages to streamline the storage and access
to archived data. 

These existing projects primarily focus on creating more useful, open, and 
accessible research products. But data- and code-intensive collaborative projects 
increasingly need collaborative solutions for data storage, sharing, and updating 
throughout the full lifecycle of the project. Furthermore, many of these data 
archive platforms focus on academic research projects and do not support some 
of the unique needs of federal reserach. 

To address future scientific data sharing and archival challenges, the USGS 
created ScienceBase (sciencebase.gov). This platform is designed to support 
the full project data lifecycle and has seen rapid adoption with USGS 
researchers and collaborators. ScienceBase supports the storage and access 
of large files and datasets. It allows data to be stored with a mixture of 
public and controlled access and has been designed from the beginning with 
first-class RESTful web interfaces to support robust API access. 

To expand the usefulness of the ScienceBase platform, we have designed and 
implemented an R interface to ScienceBase called \pkg{sbtools}. Here we 
describe unique and useful features of ScienceBase, and how we have implemented
R interfaces to make them accessible and useful from R. 

